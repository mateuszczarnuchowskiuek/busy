\chapter{Tworzenie bibliografii naukowej}
\label{chap:czwarty}



%-------------------------------------------------------
\section{Bazy dostępne w bibliotece głównej UEK}

Tworząc pracę dyplomową stajemy przed koniecznością zapoznania się szerzej z realizowanym zagadnieniem. Pomocne w tym będą bazy danych publikacji naukowych. Szeroką ofertę zapewnia Biblioteka Główna Uniwersytetu Ekonomicznego w Krakowie (\url{https://bg.uek.krakow.pl} -- opcja menu: Bazy danych), która oferuje pokaźny zbiór baz, do których dostęp posiadają zarówno studenci, jak i pracownicy Uniwersytetu Ekonomicznego w Krakowie. Korzystanie z baz danych możliwe jest z terenu kampusu uczelni. Do sporej części baz danych zapewniony jest również dostęp zdalny, poza kampusem.

W skład zbioru wchodzą zarówno bazy danych krajowe, jak i międzynarodowe. Ich lista obejmuje w szczególności: 

\begin{itemize}
	\item BazEkon,
	\item Springer Link (pełne teksty artykułów, czy książek),
	\item Scopus,
	\item Web of Science.
	\item ACM Digital Library,
	\item EBSCO,
	\item JSTOR.
\end{itemize}

Jedną z nich jest BazEkon (\url{https://bazekon.uek.krakow.pl/}), współtworzona przez Bibliotekę Główną Uniwersytetu Ekonomicznego w Krakowie. W bazie znajdują się opisy bibliograficzne artykułów z periodyków naukowych i gospodarczych, naukowych serii wydawniczych uczelni ekonomicznych, wydziałów ekonomicznych i wydziałów zarządzania uniwersytetów, a także instytucji naukowych, również pozarządowych. Znaczna część artykułów dostępna jest w pełnej wersji. Baza umożliwia wyszukiwanie danych wg autora, tytułu, źródła, czy dowolnego terminu. Istnieje możliwość wysłania wyników wyszukiwania na e-mail.

W przypadku baz międzynarodowych, na uwagę zasługuje Springer Link zawierająca pełne wersje czasopism i książek opublikowane przez koncern wydawniczy Springer Verlag oraz Kluwer Academic Publishers. Serwis oferuje publikacje m.in. z zakresu: ekonomii, biznesu, chemii, fizyki, matematyki i informatyki, statystyki, prawa oraz medycyny. Dostęp do bazy możliwy jest dzięki ogólnokrajowej licencji w całości finansowanej przez Ministerstwo Nauki i Szkolnictwa Wyższego.



%-------------------------------------------------------
\section{Baza Google Scholar}

Ogólnodostępną bazą danych publikacji naukowych jest również Google Scholar. Korzystanie z niej jest bardzo proste. Należy:

\begin{itemize}
	
	\item przejść do witryny Google Scholar (\url{https://scholar.google.pl/}),
	
	\item wprowadzić ciąg do wyszukiwania, np. dla uzyskania informacji o publikacjach naukowych, które dotyczą zalet i wad wykorzystania e-learningu można wprowadzić:\\
	\textbf{e-learning zalety wady} lub też \textbf{e-learning advantages disadvantages},
	
	\item zawęzić wyniki wyszukiwania (np. do publikacji z ostatnich kilku lat lub ustalić inne kryteria filtrowania),
	
	\item przejrzeć odszukane publikacje; zapoznać się z ich opisem,
	
	\item pobrać wersję elektroniczną publikacji (dla sporej liczby publikacji dostępna jest wersja elektroniczna w formatach PDF, HTML, itp.).
	
\end{itemize}

W przypadku chęci powołania się na odszukaną w bazie publikację należy skorzystać z symbolu znaku cudzysłowu znajdującego się w ostatnim wierszu opisu publikacji. Kliknięcie w ten symbol spowoduje wyświetlenie opisu bibliograficznego publikacji w kilku powszechnie używanych formatach. Należy skopiować opis w formacie akceptowanym przez narzędzie edycyjne używane do sporządzenia pracy dyplomowej\footnotemark. Następnie powinno się przywołać tę publikację (zacytować ją) w tekście swojej pracy dyplomowej.

\footnotetext{W przypadku użycia do składu pracy dyplomowej języka \LaTeX~należy wykorzystać opis bibliograficzny w formacie BibTeX.}

