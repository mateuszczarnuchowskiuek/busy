\chapter*{Wstęp}
\label{chap:wstep}
\addcontentsline{toc}{chapter}{Wstęp}




Niniejszy ,,Przewodnik dyplomanta'' został opracowany dla zapewnienia najwyższej jakości prac dyplomowych realizowanych przez studentów Uniwersytetu Ekonomicznego w Krakowie. Zawiera on zbiór zasad, których zastosowanie pozwoli w znacznym stopniu usprawnić proces tworzenia pracy.

Pracę dyplomową można przygotować przy użyciu dowolnego narzędzia przeznaczonego do edycji tekstu. Zachęcamy jednak, aby posłużyć się oprogramowaniem do profesjonalnego, zautomatyzowanego składu tekstu \TeX~ \citep{website:latex}, dla którego dostosowaliśmy istniejący już szablon pracy dyplomowej w Uniwersytecie Ekonomicznym w Krakowie. Twórcą pierwotnej wersji szablonu jest \textit{Maciej Sypień}. Zastosowanie narzędzia \TeX~wraz z opracowanym szablonem pozwoli na szybkie i komfortowe przygotowanie pracy dyplomowej, umożliwiając uzyskanie końcowego dokumentu o profesjonalnym wyglądzie. Niniejszy ,,Przewodnik dyplomanta'' został wykonany właśnie przy użyciu narzędzia \TeX~oraz opracowanego szablonu. Opracowaliśmy również szablon pracy dyplomowej dla edytora tekstu Microsoft Word dla osób, które preferują korzystanie z tego narzędzia.



\begin{flushright}
~\\
\textit{Autorzy} \\
~\\
Janusz Stal \\
Grażyna Paliwoda-Pękosz\\
Katedra Informatyki\\
Uniwersytetu Ekonomicznego w Krakowie\\

\end{flushright}

