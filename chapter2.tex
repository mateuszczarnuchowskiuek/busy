\chapter{Struktura pracy dyplomowej}
\label{chap:drugi}



%-------------------------------------------------------
\section{Wstęp}

Wstęp pracy dyplomowej powinien zawierać ogólny zarys i tło badanego problemu oraz przesłanki dla podjęcia realizowanego tematu. Ponadto we wstępie należy jasno sformułować cel i zakres pracy, pytania badawcze lub hipotezy badawcze oraz scharakteryzować krótko sposób realizacji celu pracy. Należy również przedstawić skrótowo, co będzie przedmiotem poszczególnych rozdziałów pracy.


%-------------------------------------------------------
\section{Rozdziały}
 Początkowe rozdziały pracy (jeden do dwóch) zawierają definicje podstawowych pojęć, kluczowych dla tematu pracy oraz przegląd aktualnej literatury związanej z tematem pracy.
 Kolejny rozdział powinien zawierać opis metody badawczej zastosowanej do zrealizowania celu pracy, odpowiedzi na pytania badawcze lub weryfikacji hipotez badawczych. Ostatnie rozdziały pracy (jeden do dwóch) stanowią zwykle wkład własny dyplomanta w realizację celu pracy \citep{blazejewskiSzal}.

Cel pracy może zostać zrealizowany poprzez  badania empiryczne, eksperymentalne, studium przypadku, zaprojektowanie aplikacji użytkowej, czy opracowanie teoretycznego modelu. Możliwe jest również przeprowadzenie systematycznego przeglądu literatury. Poniżej zostały krótko scharakteryzowane wymienione metody.

\textbf{Badania empiryczne} to rodzaj badań, które za źródło danych potrzebnych do ich wykonania uznają wiedzę zdobytą w wyniku ukierunkowanych obserwacji lub eksperymentów. Ich celem jest uzyskanie odpowiedzi na pytanie badawcze sformułowane przez dyplomanta. Możliwe jest wyróżnienie następujących etapów:

\begin{itemize}
	\item zdefiniowanie pytań (ankiety),
	\item implementacja ankiety (można to zrealizować korzystając z Formularzy Google),
	\item przeprowadzenie badań pilotażowych (sprawdzenie poprawności założonej procedury badawczej: doboru badanych osób, przyjętych wskaźników zmiennych, czy użytych narzędzi badawczych),
	\item realizacja właściwych badań skierowanych do docelowej grupy respondentów,
	\item przedstawienie wyników ankiety, 
	\item analiza/dyskusja wyników ankiety, odpowiedź na pytanie badawcze,
	\item sporządzenie wniosków/podsumowania.
\end{itemize}

\textbf{Badania eksperymentalne} \citep{kopczewski2007ekonomia, brzezinski2015badania} są przykładem metod ilościowych. Polegają na stworzeniu przestrzeni badawczej dla prowadzonych eksperymentów. Zwykle ograniczają się one do wygenerowania zestawu lub zestawów danych, które będą następnie przetwarzane przez określone warianty algorytmów czy metod. Otrzymane w ten sposób wyniki stanowią podstawę dla dokonania porównań oraz realizacji założonych celów badawczych czy weryfikacji postawionych tez.

\textbf{Studium przypadku} (ang. case study) to metoda badawcza zawierająca szeroki opis danego zjawiska, mająca na celu jego pogłębioną analizę i ocenę \citep{baxter2008qualitative, crowe2011case}. Możliwe jest wyróżnienie następujących jej etapów:

\begin{itemize}
	\item znalezienie ,,przypadku'' do analizy, w którym zachodzą procesy powiązane z realizowanym tematem pracy; może to być firma, urząd, … - tj. jednostka, do której dyplomant posiada dostęp (np. ze względu na swoją pracę),
	\item zrealizowanie wywiadów z pracownikami jednostki, opartych na uprzednio przygotowanych pytaniach, a także przegląd materiałów jednostki oraz materiałów z innych źródeł mających związek z tematem pracy,
	\item przedstawienie wyników (opis jednostki oraz zagadnień związanych z tematem pracy),
	\item analiza/dyskusja wyników, odpowiedź na pytanie(a) badawcze,
	\item sporządzenie wniosków/podsumowania.
\end{itemize}

W ramach pracy dyplomowej możliwe jest także opracowanie i \textbf{wykonanie aplikacji użytkowej} służącej do rozwiązania zdefiniowanego problemu. Praca dyplomowa stanowić będzie opis wykonanego programu, będąc ilustracją/rozwiązaniem problemu scharakteryzowanego w początkowych rozdziałach pracy.

Praca dyplomowa może dotyczyć również \textbf{opracowania modelu teoretycznego}, powstałego na bazie przeglądu literatury przedstawionego w początkowych rozdziałach pracy.

\textbf{Systematyczny przegląd literatury} \citep{mazur2018jak} umożliwia udzielenie odpowiedzi na pytanie badawcze za pomocą jawnych i ściśle określonych sposobów identyfikacji oraz oceny i syntezy źródeł. Można wyróżnić następujące etapy przygotowania przeglądu systematycznego:

\begin{itemize}
	\item określenie metody badawczej (w tym: (1) określenie baz bibliograficznych, które będą brane pod uwagę (Web of Science, Scopus, Google scholar, Springer Link, ACM Digital Library, EBSCO, JSTOR, …), (2) zdefiniowanie słów kluczowych, według których odbędzie się przeszukiwanie bazy danych, (3) określenie zakresu czasowego przeszukiwań bazy danych (zakres lat) oraz rodzaju źródeł (książki, artykuły,…)),
	\item przedstawienie wyników wyszukiwań (liczba pozycji pojawiających się jako wyniki wyszukiwań, liczba ,,odrzuconych'' pozycji, liczba pozycji poddanych dalszej analizie, zakres zagadnień poruszanych w badanych publikacjach),
	\item analiza/dyskusja wyników, odpowiedź na pytanie(a) badawcze, 
	\item sporządzenie wniosków/podsumowania.
\end{itemize}

Wymienione metody nie wyczerpują możliwości realizacji celu pracy. 
Możliwe jest zastosowanie innego podejścia w porozumieniu z opiekunem naukowym.


%-------------------------------------------------------
\section{Zakończenie}

W zakończeniu należy dokonać podsumowania, odnosząc się do stawianych na wstępie celów pracy oraz sformułować odpowiedzi na zdefiniowane pytania badawcze lub omówić wyniki weryfikacji hipotez badawczych. Powinno się  streścić zawartość każdego z rozdziałów pracy (jeden akapit streszczenia dla każdego z rozdziałów).


%-------------------------------------------------------
\section{Wykaz rysunków i tablic}
Wykaz rysunków i tablic powinien zostać sporządzony automatycznie z wykorzystaniem dostępnych narzędzi (zob. Załącznik w przypadku korzystania z szablonu pracy w Latex lub narzędzia MS Word).


%-------------------------------------------------------
\section{Bibliografia}

Bibliografia zawiera spis prac, które zostały wykorzystane w redagowanym dokumencie (dla których istnieje cytowanie w tekście dokumentu). Spis prac powinien zostać uporządkowany alfabetycznie. Należy sporządzić go w oparciu o zestaw reguł APA (American Psychological Association), który jest jednym z najczęściej stosowanych w cytowaniu źródeł w naukach społecznych. Dostępne są liczne opracowania dotyczące zasad formatowania APA \citep{website:apa} dla książek \citep{weglinska}, artykułów w czasopismach \citep{barczak-brezinski}, źródeł internetowych \citep{website:paragraph}, czy materiałów konferencyjnych \citep{saltz2019exploring}. Należy zwrócić uwagę na użycie w cytowaniach wyłącznie nazwisk, bez imienia (inicjału imienia) autora.

Dla przywołania pracy wymienionej w bibliografii w tekście redagowanego dokumentu należy również stosować zestaw reguł APA. Nie należy wykorzystywać do tego celu przypisów dolnych. Każda praca wymieniona w bibliografii powinna zostać przywołana (zacytowana) w tekście redagowanego dokumentu przynajmniej jednokrotnie. W przypadku dosłownego cytowania, lub powoływania rysunku, czy tabeli, należy podać również numer strony, gdzie cytowany tekst występuje, zgodnie ze specyfikacją APA\footnotemark.

\footnotetext{Dla języka LaTeX sposoby cytowania pozycji z bibliografii z wykorzystaniem biblioteki natbib zaprezentowano w \citep{website:wikibooks}.}

Pożądane jest, aby do obsługi cytowań stosować, dostępne w edytorach tekstu, narzędzia służące do zarządzania bibliografią. W przypadku użycia \LaTeX, przykładowy wykaz komend dla cytowania wewnątrztekstowego podany został tabeli \ref{tab:bnatbib}. 


\begin{table}[ht]
	\centering
	\caption{Przykładowy zestaw komend dla cytowania wewnątrztekstowego przy użyciu narzędzia \LaTeX~oraz pakietu natbib}
	\begin{tabularx}{\textwidth}{l l}
		\hline
		\textbf{Komenda} & \textbf{Efekt} \\
		\hline
		\textbackslash citet\{goossens93\}			& Goossens et al. (1993)\\
		\textbackslash citep\{goossens93\}			& (Goossens et al., 1993)\\
		\textbackslash citet*\{goossens93\}			& Goossens, Mittlebach, and Samarin (1993)\\
		\textbackslash citep*\{goossens93\}			& (Goossens, Mittlebach, and Samarin, 1993)\\
		\textbackslash citeauthor\{goossens93\}		& Goossens et al.\\
		\textbackslash citeauthor*\{goossens93\}	& Goossens, Mittlebach, and Samarin\\
		\textbackslash citeyear\{goossens93\}		& 1993\\
		\textbackslash citeyearpar\{goossens93\}	& (1993)\\
		\textbackslash citealt\{goossens93\}		& Goossens et al. 1993\\
		\textbackslash citealp\{goossens93\}		& Goossens et al., 1993\\
		\hline		
	\end{tabularx}
	\caption*{Źródło: opracowanie własne na podstawie: \citep{website:wikibooks}.}
	\label{tab:bnatbib}
\end{table}


Bibliografia, w przypadku pracy dyplomowej, powinna zawierać kilkanaście do kilkudziesięciu pozycji, z którymi student dokładnie się zapoznał i wykorzystał je w redagowanym dokumencie. Zgodnie z wymogami Uniwersytetu, co najmniej dwie pozycje bibliograficzne muszą dotyczyć publikacji wydanych w języku angielskim. Należy opierać się przede wszystkim na pracach, które zostały poddane recenzji, a następnie wydane (książki, artykuły w czasopismach), do minimum ograniczając źródła internetowe o wątpliwej jakości.