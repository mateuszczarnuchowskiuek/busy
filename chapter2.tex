\chapter{Model biznesowy}
\label{chap:drugi}



%-------------------------------------------------------
\section{Propozycja wartości}

Propozycja wartości projektu polega na usprawnieniu funkcjonowania lokalnego transportu realizowanego przez prywatnych przewoźników. Ta propozycja wartości zawiera w sobie dwie zazębiające się propozycje wartości skierowane do dwóch różnych grup: propozycję wartości dla pasażerów oraz propozycję wartości dla przewoźników.

Propozycja wartości dla pasażerów polega na tym, że lokalny transport w przeciwieństwie do transportu zbiorowego w dużych miastach, czy transportu dalekobieżnego jest bardzo rozdrobniony i opiera się na małych konkurujących ze sobą prywatnych przewoźnikach. Przez to rozdrobnienie planowanie podróży jest mocno utrudnione. Przykładowo przewoźnicy publikują swoje rozkłady jazdy w wielu różnych miejscach, co utrudnia zorientowanie się przez pasażerów o wszystkich możliwych kursach, a zakup biletów odbywa się zwykle fizycznie w pojazdach, co jest mniej wygodne dla pasażerów niż zakup biletu w aplikacji lub przez stronę internetową oraz niepotrzebnie zabiera czas i uwagę kierowcy. Planowane przedsięwzięcie rozwiązywałoby te problemy tworząc jedną bazę rozkładów jazdy oraz umożliwiając wcześniejszy zakup biletu za pośrednictwem platformy. Oprócz tego mali przewoźnicy ze względu na swoje ograniczone zasoby zazwyczaj nie posiadają systemu do śledzenia lokalizacji pojazdów na żywo. Taki natomiast bardzo ułatwia pasażerom planowanie podróży, ponieważ pozwala się im zorientować o ewentualnych opóźnieniach względem rozkładu jazdy lub też sprawdzić, czy ich bus już odjechał z danego przystanku czy jeszcze nie. Planowana platforma dostarczałaby podróżnym właśnie takich danych.

Propozycja wartości dla przewoźników polega natomiast na tym, że ich kursy byłyby agregowane w przystępnej dla podróżnych formie, co może zwiększyć ilość pasażerów, a w konsekwencji zyski oraz pomoże marketingowo konkurować z większymi przewoźnikami poprzez zwiększenie potencjału do odkrycia ich oferty przez podróżnych. Do tego kolejną rzeczą, która może zwiększyć ilość pasażerów jest możliwość łatwiejszego zakupu biletów za pośrednictwem platformy. Ta możliwość wpłynie również pozytywnie na realizację kursów, ponieważ kierowcy będą musieli mniej czasu i uwagi poświęcać na fizyczną sprzedaż biletów w pojeździe na każdym kolejnym przystanku. Jak wspomniano wyżej, mali przewoźnicy zazwyczaj nie posiadają własnych systemów do śledzenia lokalizacji pojazdów ze względu na ograniczone środki. Tutaj jako rozwiązanie planowana jest aplikacja dedykowana dla kierowców, która zamieni telefon kierowcy w dashboard dla pojazdu, który będzie wyświetlał parametry kursu oraz jednocześnie będzie nadawał swoją pozycję na żywo. Pozwoli to w bardzo łatwy i niemal bezkosztowy dla przewoźnika sposób na dołączenie do systemu pojazdów nie posiadających nadajników GPS.

\section{Model przychodów}

Lorem ipsum dolor sit amet, consectetur adipiscing elit, sed do eiusmod tempor incididunt ut labore et dolore magnam aliquam quaerat voluptatem. Ut enim aeque doleamus animo, cum corpore dolemus, fieri tamen permagna accessio potest, si aliquod aeternum et infinitum impendere malum nobis opinemur. Quod idem licet transferre in voluptatem, ut postea variari voluptas distinguique.

\section{Rynek}
\subsection{Planowana przestrzeń rynkowa firmy i potencjalne możliwości finansowe}
Platforma ma na celu integrację lokalnych przewoźników, w tym mikro- i małych przedsiębiorstw oferujących usługi przewozowe zarówno na krótkich, jak i długich dystansach. Planowana przestrzeń rynkowa obejmuje powiaty oraz regiony o rozproszonej infrastrukturze transportowej, gdzie istnieje znaczne zapotrzebowanie na usługi przewozowe, a jednocześnie brak jest kompleksowego rozwiązania, które łączyłoby różne lokalne firmy w jedną spójną sieć.
W Polsce sektor transportu osób, obejmujący przewozy lokalne i długodystansowe, dynamicznie się rozwija, szczególnie w regionach, gdzie transport publiczny jest niewystarczający lub niedostatecznie dostępny. Wartość rynku transportu osobowego w Polsce szacuje się na około 8–10 miliardów złotych, z czego znaczna część dotyczy właśnie przewoźników lokalnych, którzy działają indywidualnie lub w niewielkich grupach.

Nasza platforma ma potencjał do zmniejszenia rozdrobnienia tego rynku, oferując wspólną przestrzeń cyfrową, która ułatwi klientom dostęp do usług przewozowych, a przewoźnikom umożliwi efektywne zarządzanie zleceniami i zwiększenie liczby klientów.

Rynek przewozów osobowych w Polsce rozwija się w szybkim tempie, głównie dzięki rosnącej popularności cyfrowych narzędzi umożliwiających zamawianie transportu. Wartość sektora przewozów osobowych dynamicznie rośnie, szczególnie w regionach, gdzie transport publiczny jest niewystarczający. Nasza platforma ma potencjał, by zniwelować istniejące bariery poprzez łatwą dostępność i elastyczność oferty.

\subsection{Obszar rynku, który firma zamierza obsługiwać}
Platforma będzie działać w następujących segmentach:
\begin{enumerate}
     
    \item Przewozy lokalne w obrębie powiatów
        Obsługa krótkodystansowych tras w granicach powiatów, takich jak transport do pracy, szkoły, czy na zakupy.
        Skierowanie oferty do mieszkańców miast średniej wielkości oraz mniejszych miejscowości, gdzie transport publiczny jest ograniczony.

    \item Przewozy długodystansowe
        Realizacja przewozów międzymiastowych lub międzyregionowych dla klientów indywidualnych.
        Zwiększenie elastyczności podróży na większe odległości dzięki współpracy z lokalnymi przewoźnikami w ramach jednej platformy.
 
\end{enumerate}      

\subsection{Rozmiar rynku i jego podział na nisze}
    Rynek przewozów osobowych można podzielić na dwa kluczowe segmenty:
    \begin{enumerate}
        \item Przewozy krótkodystansowe = 60\%
            Zapotrzebowanie na przejazdy w granicach jednego powiatu lub regionu, gdzie klienci poszukują wygodnych i dostępnych usług transportowych na krótkie dystanse.
        \item Przewozy długodystansowe - 40\% rynku
            Przejazdy międzymiastowe i międzyregionowe, które umożliwiają klientom wygodną i konkurencyjną cenowo alternatywę wobec większych firm przewozowych lub transportu publicznego.
    \end{enumerate}

    
\section{Otoczenie konkurencyjne}

    \subsection{Bezpośredni konkurenci}
    Naszymi bezpośrednimi konkurentami są platformy takie jak Jakdojade oraz e-podróżnik.pl, które oferują integrację informacji o połączeniach transportowych. Jakdojade koncentruje się głównie na rozkładach jazdy komunikacji miejskiej, natomiast e-podróżnik obejmuje również większe odległości, integrując informacje o trasach autobusowych, kolejowych i autokarowych. Obie platformy mają znaczącą obecność na rynku, jednak skupiają się głównie na dużych przewoźnikach i publicznym transporcie masowym.

    \subsection{Pośredni konkurenci}
    Aplikacje własne dużych przewoźników, takich jak PKP Intercity, FlixBus czy regionalni operatorzy autobusowi, są istotnym elementem rynku. Pozwalają one na planowanie i zakup biletów w ramach konkretnego operatora, lecz nie oferują możliwości kompleksowego porównania wielu przewoźników, szczególnie w przypadku mniejszych lokalnych firm.

    \subsection{Substytuty}
    Substytutami naszej platformy są tradycyjne metody wyszukiwania i planowania podróży, takie jak przeglądanie stron internetowych przewoźników, korzystanie z rozkładów jazdy dostępnych na dworcach czy też telefoniczne rezerwacje u lokalnych przewoźników. Choć te metody wciąż są używane, są one znacznie mniej wygodne i ograniczają dostęp do pełnego obrazu rynku przewozowego.

    \subsection{Nowe podmioty na rynku}
    Pojawienie się nowych platform integrujących przewoźników jest możliwe, szczególnie w odpowiedzi na rosnące zapotrzebowanie na wygodne i szybkie rozwiązania w planowaniu podróży. W szczególności mogą pojawić się rozwiązania wspierające transport w mniejszych regionach, jednak obecne platformy mają silnie ugruntowaną pozycję na rynku, co ogranicza ich potencjalne tempo ekspansji.

    \section{Przewaga konkurencyjna}

    Nasza platforma dostarcza klientom unikalną wartość poprzez umożliwienie łatwego planowania podróży, szczególnie w regionach, gdzie transport publiczny i dostęp do lokalnych przewoźników jest ograniczony. Klienci mogą korzystać z jednej platformy, by znaleźć i porównać trasy oferowane przez mikro- i średnich przewoźników.

    Lepszy dostęp do lokalnych przewoźników
    W przeciwieństwie do dużych platform, takich jak Jakdojade czy e-podróżnik.pl, nasza usługa koncentruje się na obsłudze połączeń regionalnych i długodystansowych realizowanych przez mniejsze firmy, które są często niedostępne w innych systemach. Klienci zyskują lepszą elastyczność i dostęp do mniej popularnych tras.

    Redukcja barier informacyjnych
    Dzięki naszej platformie klienci nie muszą przeszukiwać wielu źródeł, by znaleźć odpowiednie połączenie. Zapewniamy zintegrowane, przejrzyste informacje o trasach, godzinach i cenach, co znacząco upraszcza proces planowania podróży.

\section{Strategia rynkowa}

    \subsection{Plan wejścia na rynek}
    Nasza strategia zakłada stopniowe wejście na rynek, zaczynając od wybranych regionów powiatowych, gdzie brak zintegrowanych usług przewozowych. Po przetestowaniu i udoskonaleniu platformy planowana jest ekspansja na kolejne powiaty i regiony. Kluczowym elementem strategii jest równoległe budowanie bazy użytkowników i przewoźników.

    \subsection{Promocja w mediach społecznościowych}
    Aby dotrzeć do klientów indywidualnych, firma skoncentruje się na:
    \begin{enumerate}
        \item Reklamach w mediach społecznościowych – kampanie w serwisach takich jak Facebook i Instagram, które będą promować łatwość korzystania z platformy oraz jej dostępność w regionach lokalnych.
        \item Programie poleceń – użytkownicy będą zachęcani do zapraszania znajomych i przewoźników poprzez system nagród, co zwiększy bazę użytkowników.
    \end{enumerate}

    \subsection{Bezpośredni kontakt z przewoźnikami}
    Kluczowym elementem strategii będzie osobisty kontakt z mikro- i średnimi przewoźnikami, aby przekonać ich do współpracy. Planowane działania obejmują:
    \begin{enumerate}
        \item Spotkania z właścicielami firm przewozowych, by wyjaśnić korzyści płynące z dołączenia do platformy.
        \item Przekazywanie jasnych informacji o możliwościach zwiększenia obłożenia tras i pozyskiwania nowych klientów.
    \end{enumerate}

\section{Projekt organizacji}

    \subsection {Plan organizacji pracy na początkowym etapie}
    Na starcie działalności firma zatrudni niewielki zespół, którego zadaniem będzie obsługa kluczowych obszarów operacyjnych. Zespół będzie podzielony na dwa główne obszary:
    \begin{enumerate}
        \item Dział marketingu, obsługi klienta i kontaktu z przewoźnikami zajmujący się promowaniem platformy, wsparciem klientów indywidualnych oraz nawiązywaniem współpracy z lokalnymi przewoźnikami.
        \item Zespół developerów odpowiedzialny za rozwój i utrzymanie aplikacji, zapewniając jej płynne funkcjonowanie i wdrażanie nowych funkcji.
    \end{enumerate}
    
    \subsection {Struktura organizacyjna docelowa}
    W miarę rozwoju firmy i wzrostu liczby użytkowników oraz przewoźników, struktura organizacyjna będzie skalowana w następujący sposób:
    \begin{enumerate}
        \item Początkowy zespół developerów zostanie rozszerzony do pełnoprawnego działu IT, odpowiedzialnego za rozwój platformy, wsparcie techniczne i wdrażanie nowych funkcji.
        \item Powstanie samodzielny zespół marketingowy, który będzie zarządzać kampaniami reklamowymi, mediami społecznościowymi oraz programem poleceń.
        \item Zespół Client Service rozwinie się do dedykowanuch zespołów ds. obsługi użytkowników i ds. obsługi przewoźników. 
        \begin{itemize}
            \item Zespół ds. obsługi użytkowników będzie odpowiedzialny za wsparcie klientów platformy.
            \item Dedykowane osoby ds. obsługi przewoźników będą odpowiedzialne za zarządzanie relacjami z partnerami przewozowymi, wsparcie techniczne i operacyjne.
            \end{itemize}
    \end{enumerate}
\section{Zespół menedżerski}


\subsection{Rola zespołu zarządzającego}
Zespół zarządzający odpowiada za nadzór nad kluczowymi obszarami działalności firmy: technologią, marketingiem oraz relacjami z przewoźnikami. Jego celem jest zapewnienie sprawnego wdrożenia i rozwoju platformy oraz budowanie trwałych relacji z klientami i partnerami biznesowymi.

\subsection{Kluczowe obszary działalności i liderzy zespołu}
\begin{itemize}
    \item \textbf{Technologia (IT)}\\
    Lider zespołu IT odpowiada za rozwój i stabilność platformy. Do jego obowiązków należy zarządzanie zespołem developerów, wdrażanie nowych funkcji oraz zapewnienie, że platforma działa sprawnie i odpowiada na potrzeby użytkowników.\\
    \textit{Kompetencje:} doświadczenie w zarządzaniu projektami IT, znajomość technologii webowych i mobilnych, umiejętność szybkiego rozwiązywania problemów technologicznych.
    
    \item \textbf{Marketing}\\
    Lider marketingu kieruje działaniami promocyjnymi, tworzy strategie reklamowe i kampanie w mediach społecznościowych, a także rozwija programy poleceń. Jego celem jest budowanie rozpoznawalności marki i pozyskiwanie nowych użytkowników.\\
    \textit{Kompetencje:} znajomość strategii marketingu cyfrowego, umiejętność analizy rynku i planowania działań promocyjnych, doświadczenie w tworzeniu treści reklamowych.
    
    \item \textbf{Relacje z przewoźnikami (Client Relations)}\\
    Lider relacji z przewoźnikami odpowiada za współpracę z mikro- i średnimi firmami przewozowymi. Jego zadaniem jest pozyskiwanie nowych partnerów, budowanie długoterminowych relacji oraz wsparcie w zakresie korzystania z platformy.\\
    \textit{Kompetencje:} doskonałe zdolności komunikacyjne, umiejętność negocjacji, znajomość rynku przewozowego oraz potrzeb małych przewoźników.
\end{itemize}

\subsection{Synergia zespołu zarządzającego}
Każdy lider odpowiada za swój obszar działalności, ale kluczowe decyzje podejmowane są wspólnie, aby zapewnić spójność działań i skuteczne wdrożenie strategii biznesowej. Dzięki połączeniu kompetencji technologicznych, marketingowych i relacyjnych zespół zarządzający jest w stanie efektywnie zarządzać platformą, dostosowywać się do zmian na rynku i rozwijać ofertę w odpowiedzi na potrzeby klientów.