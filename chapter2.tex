\chapter{Model biznesowy}
\label{chap:drugi}



%-------------------------------------------------------
\section{Propozycja wartości}

Lorem ipsum dolor sit amet, consectetur adipiscing elit, sed do eiusmod tempor incididunt ut labore et dolore magnam aliquam quaerat voluptatem. Ut enim aeque doleamus animo, cum corpore dolemus, fieri tamen permagna accessio potest, si aliquod aeternum et infinitum impendere malum nobis opinemur. Quod idem licet transferre in voluptatem, ut postea variari voluptas distinguique.

\section{Model przychodów}

Lorem ipsum dolor sit amet, consectetur adipiscing elit, sed do eiusmod tempor incididunt ut labore et dolore magnam aliquam quaerat voluptatem. Ut enim aeque doleamus animo, cum corpore dolemus, fieri tamen permagna accessio potest, si aliquod aeternum et infinitum impendere malum nobis opinemur. Quod idem licet transferre in voluptatem, ut postea variari voluptas distinguique.

\section{Rynek}
\subsection{Planowana przestrzeń rynkowa firmy i potencjalne możliwości finansowe}
Platforma ma na celu integrację lokalnych przewoźników, w tym mikro- i małych przedsiębiorstw oferujących usługi przewozowe zarówno na krótkich, jak i długich dystansach. Planowana przestrzeń rynkowa obejmuje powiaty oraz regiony o rozproszonej infrastrukturze transportowej, gdzie istnieje znaczne zapotrzebowanie na usługi przewozowe, a jednocześnie brak jest kompleksowego rozwiązania, które łączyłoby różne lokalne firmy w jedną spójną sieć.
W Polsce sektor transportu osób, obejmujący przewozy lokalne i długodystansowe, dynamicznie się rozwija, szczególnie w regionach, gdzie transport publiczny jest niewystarczający lub niedostatecznie dostępny. Wartość rynku transportu osobowego w Polsce szacuje się na około 8–10 miliardów złotych, z czego znaczna część dotyczy właśnie przewoźników lokalnych, którzy działają indywidualnie lub w niewielkich grupach.

Nasza platforma ma potencjał do zmniejszenia rozdrobnienia tego rynku, oferując wspólną przestrzeń cyfrową, która ułatwi klientom dostęp do usług przewozowych, a przewoźnikom umożliwi efektywne zarządzanie zleceniami i zwiększenie liczby klientów.

Rynek przewozów osobowych w Polsce rozwija się w szybkim tempie, głównie dzięki rosnącej popularności cyfrowych narzędzi umożliwiających zamawianie transportu. Wartość sektora przewozów osobowych dynamicznie rośnie, szczególnie w regionach, gdzie transport publiczny jest niewystarczający. Nasza platforma ma potencjał, by zniwelować istniejące bariery poprzez łatwą dostępność i elastyczność oferty.

\subsection{Obszar rynku, który firma zamierza obsługiwać}
Platforma będzie działać w następujących segmentach:
\begin{enumerate}
     
    \item Przewozy lokalne w obrębie powiatów
        Obsługa krótkodystansowych tras w granicach powiatów, takich jak transport do pracy, szkoły, czy na zakupy.
        Skierowanie oferty do mieszkańców miast średniej wielkości oraz mniejszych miejscowości, gdzie transport publiczny jest ograniczony.

    \item Przewozy długodystansowe
        Realizacja przewozów międzymiastowych lub międzyregionowych dla klientów indywidualnych.
        Zwiększenie elastyczności podróży na większe odległości dzięki współpracy z lokalnymi przewoźnikami w ramach jednej platformy.
 
\end{enumerate}      

\subsection{Rozmiar rynku i jego podział na nisze}
    Rynek przewozów osobowych można podzielić na dwa kluczowe segmenty:
    \begin{enumerate}
        \item Przewozy krótkodystansowe = 60\%
            Zapotrzebowanie na przejazdy w granicach jednego powiatu lub regionu, gdzie klienci poszukują wygodnych i dostępnych usług transportowych na krótkie dystanse.
        \item Przewozy długodystansowe - 40\% rynku
            Przejazdy międzymiastowe i międzyregionowe, które umożliwiają klientom wygodną i konkurencyjną cenowo alternatywę wobec większych firm przewozowych lub transportu publicznego.
    \end{enumerate}

    
\section{Otoczenie konkurencyjne}

\subsection{Bezpośredni konkurenci}
Naszymi bezpośrednimi konkurentami są platformy takie jak Jakdojade oraz e-podróżnik.pl, które oferują integrację informacji o połączeniach transportowych. Jakdojade koncentruje się głównie na rozkładach jazdy komunikacji miejskiej, natomiast e-podróżnik obejmuje również większe odległości, integrując informacje o trasach autobusowych, kolejowych i autokarowych. Obie platformy mają znaczącą obecność na rynku, jednak skupiają się głównie na dużych przewoźnikach i publicznym transporcie masowym.

\subsection{Pośredni konkurenci}
Aplikacje własne dużych przewoźników, takich jak PKP Intercity, FlixBus czy regionalni operatorzy autobusowi, są istotnym elementem rynku. Pozwalają one na planowanie i zakup biletów w ramach konkretnego operatora, lecz nie oferują możliwości kompleksowego porównania wielu przewoźników, szczególnie w przypadku mniejszych lokalnych firm.

\subsection{Substytuty}
Substytutami naszej platformy są tradycyjne metody wyszukiwania i planowania podróży, takie jak przeglądanie stron internetowych przewoźników, korzystanie z rozkładów jazdy dostępnych na dworcach czy też telefoniczne rezerwacje u lokalnych przewoźników. Choć te metody wciąż są używane, są one znacznie mniej wygodne i ograniczają dostęp do pełnego obrazu rynku przewozowego.

\subsection{Nowe podmioty na rynku}
Pojawienie się nowych platform integrujących przewoźników jest możliwe, szczególnie w odpowiedzi na rosnące zapotrzebowanie na wygodne i szybkie rozwiązania w planowaniu podróży. W szczególności mogą pojawić się rozwiązania wspierające transport w mniejszych regionach, jednak obecne platformy mają silnie ugruntowaną pozycję na rynku, co ogranicza ich potencjalne tempo ekspansji.

\section{Przewaga konkurencyjna}

Nasza platforma dostarcza klientom unikalną wartość poprzez umożliwienie łatwego planowania podróży, szczególnie w regionach, gdzie transport publiczny i dostęp do lokalnych przewoźników jest ograniczony. Klienci mogą korzystać z jednej platformy, by znaleźć i porównać trasy oferowane przez mikro- i średnich przewoźników.

Lepszy dostęp do lokalnych przewoźników
W przeciwieństwie do dużych platform, takich jak Jakdojade czy e-podróżnik.pl, nasza usługa koncentruje się na obsłudze połączeń regionalnych i długodystansowych realizowanych przez mniejsze firmy, które są często niedostępne w innych systemach. Klienci zyskują lepszą elastyczność i dostęp do mniej popularnych tras.

Redukcja barier informacyjnych
Dzięki naszej platformie klienci nie muszą przeszukiwać wielu źródeł, by znaleźć odpowiednie połączenie. Zapewniamy zintegrowane, przejrzyste informacje o trasach, godzinach i cenach, co znacząco upraszcza proces planowania podróży.

\section{Strategia rynkowa}

Lorem ipsum dolor sit amet, consectetur adipiscing elit, sed do eiusmod tempor incididunt ut labore et dolore magnam aliquam quaerat voluptatem. Ut enim aeque doleamus animo, cum corpore dolemus, fieri tamen permagna accessio potest, si aliquod aeternum et infinitum impendere malum nobis opinemur. Quod idem licet transferre in voluptatem, ut postea variari voluptas distinguique.

\section{Projekt organizacji}

Lorem ipsum dolor sit amet, consectetur adipiscing elit, sed do eiusmod tempor incididunt ut labore et dolore magnam aliquam quaerat voluptatem. Ut enim aeque doleamus animo, cum corpore dolemus, fieri tamen permagna accessio potest, si aliquod aeternum et infinitum impendere malum nobis opinemur. Quod idem licet transferre in voluptatem, ut postea variari voluptas distinguique.

\section{Zespół menedżerski}

Lorem ipsum dolor sit amet, consectetur adipiscing elit, sed do eiusmod tempor incididunt ut labore et dolore magnam aliquam quaerat voluptatem. Ut enim aeque doleamus animo, cum corpore dolemus, fieri tamen permagna accessio potest, si aliquod aeternum et infinitum impendere malum nobis opinemur. Quod idem licet transferre in voluptatem, ut postea variari voluptas distinguique.