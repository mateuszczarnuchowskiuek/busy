\chapter{Rodzaje prac dyplomowych}
\label{chap:pierwszy}



%-------------------------------------------------------
\section{Praca licencjacka}

Praca licencjacka wieńczy 3-letni okres kształcenia na poziomie studiów wyższych I stopnia (zawodowych). Obok zarysowanego aspektu teoretycznego, związanego z tematem pracy, np. dotyczącego zarządzania, ekonomii czy też finansów, powinna obejmować aspekt praktyczny, świadczący o tym, że student opanował określoną specjalność i posiada pewne umiejętności. Praca licencjacka powinna zawierać przedstawienie problemu i opis sposobu jego rozwiązania zaproponowany na podstawie nabytej wiedzy i umiejętności w trakcie studiów \citep{stoczewska}. Uzyskanie pozytywnej recenzji pracy, a także zdanie egzaminu upoważnia do posługiwania się tytułem licencjata.

\section{Praca inżynierska}
Podobnie jak praca licencjacka, praca inżynierska tworzona jest na koniec studiów inżynierskich I stopnia. Powinna ona spełniać podobne kryteria jak praca licencjacka, z naciskiem na aspekt praktyczny. Praca inżynierska posiada cel praktyczny wymagający podejścia inżynierskiego lub zastosowania środków technicznych. Uzyskanie pozytywnej recenzji pracy, a także zdanie egzaminu inżynierskiego upoważnia do posługiwania się tytułem inżyniera.

\section{Praca magisterska}
Praca magisterska wieńczy etap kształcenia na poziomie studiów wyższych II stopnia lub jednolitych studiów magisterskich. Powinna ona pokazywać, że student \textit{„nabył umiejętność stosowania metod i technik badawczych, posiada zdolność samodzielnego myślenia oraz twórczego opracowania tematu, jest w stanie w sposób logiczny, zgodnie z metodologiczną interpretacją zdobytej wiedzy wyłożyć wyniki swoich badań, wykazując przy tym poprawność stylistyczną oraz językową”} (\citealt{stoczewska}, s. 12). Ważnym aspektem pracy magisterskiej, odróżniającym ją od pracy licencjackiej, jest konieczność uwzględnienia w szerokim zakresie podłoża teoretycznego pracy, co objawia się w znacząco wyższych wymogach dotyczących wykorzystania źródeł literaturowych (zob. tabela \ref{tab:wymagania}). Praca magisterska świadczy o tym, że student \citep{zenderowski}:

\begin{itemize}
	\item posiada umiejętności  techniczne w zakresie pisania prac o charakterze naukowym (odpowiedni język i styl pracy, odsyłacze do literatury, spis literatury),
    \item potrafi korzystać z dostępnych zasobów wiedzy naukowej, w tym dokonywać selekcji źródeł według ich wiarygodności oraz ich porządkowania i przedstawiania w zrozumiałej formie,
    \item potrafi na podstawie zebranych i wyselekcjonowanych źródeł stworzyć tekst o charakterze naukowym, jednakże należy zaznaczyć, że praca magisterska nie jest pracą naukową, czyli nie jest konieczne, aby zawierała opis odkrycia naukowego, czy też nowych teorii.
\end{itemize}

Uzyskanie pozytywnej recenzji pracy, a także zdanie egzaminu dyplomowego upoważnia do posługiwania się tytułem magistra.


%-------------------------------------------------------
\section{Wymagania stawiane pracy dyplomowej}

Liczne wymagania stawiane pracy dyplomowej mają na celu zapewnienie jej należytej jakości. Porównanie wymagań merytorycznych prac dyplomowych zaprezentowano w tabeli~\ref{tab:porownanie}.

\begin{table}[ht]
	\centering
	\caption{Porównanie wymagań merytorycznych prac dyplomowych}
	\begin{tabularx}{\textwidth}{l X X}
		\hline
		\textbf{Kryterium} & \textbf{Licencjacka/Inżynieryjna} & \textbf{Magisterska}\\
		\hline
		Aspekt teoretyczny & Ujęty zwięźle  & Ujęty szerzej\\
		Aspekt praktyczny & Nacisk na aspekt praktyczny & Pożądany\\
		Problem badawczy & Zawężona problematyka &  Rozbudowana problematyka\\
		Aspekt metodolog. & Podstawowa orientacja dyplomanta w metodach badawczych danej dyscypliny naukowej & Zaawansowany stopień orientacji dyplomanta w zakresie podejścia naukowego oraz metodologii badań \\
		\hline		
	\end{tabularx}
	\caption*{Źródło: opracowanie własne na podstawie: (\citealt{dudziakZejmo}, Tabela 1, s. 15).}
	\label{tab:porownanie}
\end{table}

Dla zapewnienia najwyższej jakości prac dyplomowych w Uniwersytecie Ekonomicznym w Krakowie został określony zbiór minimalnych wymagań w odniesieniu do wszystkich rodzajów prac. Zbiór ten zawarty jest w tabeli \ref{tab:wymagania}. Podana w wymaganiach ,,1 strona'' to strona znormalizowana, zawierająca 1800 znaków, łącznie ze znakami odstępu (spacji). 


\begin{table}[ht]
	\centering
	\caption{Wymagania redakcyjne stawiane pracom dyplomowym}
	\begin{tabularx}{\textwidth}{X c c c}
		\hline
		\textbf{Rodzaj wymagania} & \textbf{Licencjacka} & \textbf{Inżynierska} & \textbf{Magisterska}\\
		\hline
		Minimalna objętość pracy & 50 stron & 50 stron & 70 stron\\
		Minimalna liczba rozdziałów & 3 & 3 & 4\\
		Minimalna objętość rozdziału & 10 stron & 10 stron & 10 stron\\
		Minimalna objętość wstępu & 1,5 strony & 1,5 strony & 1,5 strony\\
		Minimalna objętość zakończenia & 1,5 strony & 1,5 strony & 1,5 strony\\
		Minimalna liczba pozycji bibliograficznych & 20 & 20 & 40\\
		Minimalna liczba pozycji bibliograficznych w j.obcym & 2 & 2 & 5\\
		Stosowanie zasad redakcji pracy & tak & tak & tak\\
		Użycie stylu APA w bibliografii & tak & tak & tak\\
		\hline		
	\end{tabularx}
	\caption*{Źródło: opracowanie własne.}
	\label{tab:wymagania}
\end{table}

Należy zaznaczyć, iż każdy promotor (opiekun naukowy) może określić własne wymagania, które mogą być wyższe niż te, zawarte w tabeli \ref{tab:wymagania}.


