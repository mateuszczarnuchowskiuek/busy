\chapter{Wprowadzenie}
\label{chap:pierwszy}

\textit{„italic”}

Poniższy tekst prezentuje projekt przedsięwzięcia e-biznesowego w postacu platformy integrującej lokalnych przewoźników. Nie są oni bowiem obecnie zintegrowani w takim stopniu jak komunikacja miejska czy koleje, co sprawa, że planowanie podróży z ich udziałem jest znacząco mniej wygodne. Projekt przewiduje stworzenie platformy pozwalającej pasażerom na wyszukiwanie i planowanie tras z możliwością łatwego zakupu cyfrowych biletów odpowiednich przewoźników wewnątrz platformy. Z drugiej strony dla przewoźników będzie ona źródłem wielu cennych danych analitycznych pozwalających im optymalizować kursy i w konsekwencji zwiększać efektywność i zyski.

Zdecydowanym atutem tego przedsięwzięcia jest to, że wpisuje się ono w obecne światowe trendy polityczne dotyczące redukcji śladu węglowego. Jak możemy przeczytać w artykule:
\begin{quote}\textit{„With the increasing prominence of environmental protection, energy conservation, and carbon reduction issues [[ 1 ]], ensuring sustainable transportation [[ 3 ]] has become a critical and challenging goal. This issue is gaining more and more social attention, as environmental, social, governance (ESG) [[ 5 ]] assessments of companies' data and indicators are being given greater importance. Among the elements discussed, the significance of the "Environment" aligns closely with the United Nations' 2030 Agenda for sustainable development, with its 17 sustainable development goals (SDGs) [[ 7 ], [ 9 ], [11]], which concentrate on fostering "Sustainable Cities and Communities". One key aspiration within this overarching aim is to establish an accessible, affordable, and secure transportation system that can cater to the needs of all individuals by 2030, while simultaneously enhancing road safety measures. Special emphasis is dedicated to addressing the requirements of marginalized communities, including children, women, individuals facing mobility challenges, and the elderly. This will be achieved through the expansion of public transportation services, thoughtfully tailored to meet their specific demands.”} \citep{advancingESG} \end{quote}
Jako że komunikacja zbiorowa generuje mniejszy ślad węglowy na osobę w porównaniu do transportu indywidualnego, to ułatwienie planowania podróży transportem regionalnym powinno przełożyć się na większe zainteresowanie tą formą transportu, a w konsekwencji na ograniczenie szkodliwych emisji. Z racji tego, rządy wielu krajów, a w szczególności krajów Unii Europejskiej będą przychylnie patrzeć na wszelkie inicjatywy promujące transport zbiorowy, co z dużą pewnością sprawia, że otoczenie prawne tego przedsięwzięcia powinno być przyjazne i stabilne.

Kolejną zaletą tej działalności jest to, że transport publiczny jest biznesem odpornym na recesje, bo to właśnie podczas recesji zwiększa się udział ludzi korzystających z transportu publicznego, ponieważ pasażerowie wybierając transport zbiorowy szukają oszczędności w porównaniu do transportu indywidualnego.

Do tego dzięki zebranym danym platforma będzie w stanie zaoferować lokalnym przewoźnikom lepsze planowanie kursów i tras dzięki czemu będą oni w stanie zwiększyć swoją efektywność i zwiększyć zyski, co powinno się przełożyć na chęć ich udziału w tym przedsięwzięciu, a im więcej przewoźników nawiąże współpracę z platformą, tym system będzie miał więcej do zaoferowania dzięki efektowi sieci. Zbudujemy zatem silną pozycję na rynku co pozwoli nam na łatwą monetyzację.