\chapter{Wprowadzenie}
\label{chap:pierwszy}

%(zobacz rys. \ref{fig:logo})
%\begin{figure}[ht]
%	\centering
%	\includegraphics[width=50mm]{images/pipp.png}
%	\caption{Prototyp loga platformy PIPP}
%	\caption*{Źródło: opracowanie własne.}
%    \label{fig:logo}
%\end{figure}

Poniższy tekst prezentuje projekt przedsięwzięcia e-biznesowego w postaci platformy integrującej prywatnych przewoźników regionalnych. Mali lokalni przewoźnicy nie są bowiem obecnie zintegrowani w takim stopniu jak transport zbiorowy w miastach, koleje czy komunikacja dalekobieżna, co sprawa, że planowanie podróży z ich udziałem jest znacząco mniej wygodne dla podróżnych. Projekt przewiduje stworzenie platformy pozwalającej pasażerom na wyszukiwanie i planowanie tras z możliwością łatwego zakupu cyfrowych biletów wewnątrz platformy podobnej do istniejących platform stosowanych przykładowo przez przewoźników kolejowych. \citep{Drewnowski_Małachowski_2018} Dodatkowo podróżni zyskają dostęp do rozkładów jazdy i śledzenia nadawanej na żywo lokalizacji pojazdów różnych przewoźników w jednym miejscu. Z drugiej strony dla firm transportowych platforma będzie źródłem wielu cennych danych analitycznych pozwalających im optymalizować trasy i kursy, a w konsekwencji zwiększać efektywność i zyski. Obecnie bowem ze względu na rozdrobnienie rynku przewozowego takie dane trudno pozyskać.
\begin{quote}
    \textit{„Wyraźnie brakuje wyczerpujących i bogatych w aktualne dane opracowań dotyczących szerzej pojętej tematyki autobusowych przewozów pozamiejskich. Jedną z głównych tego przyczyn jest postępujące rozdrobnienie rynku przewozowego, wywołujące trudności m.in. w prostej inwentaryzacji danych służących dalszym analizom.”} (\citealt{Wolański_Mrozowski_Pieróg_2016}, s. 64)
\end{quote}
Dostęp do takich bez wątpienia cennych danych powinnien się przełożyć na chęć udziału przewoźników w tym przedsięwzięciu, a dodatkowo im więcej przewoźników nawiąże współpracę z platformą, tym system będzie miał więcej do zaoferowania dzięki efektowi sieci. Tym sposobem im więcej firm nawiąże współpracę z platformą tym więcej firm będzie chciało nawiązać współpracę z platformą.

Zdecydowanym atutem tego przedsięwzięcia jest także to, że wpisuje się ono w obecne światowe trendy polityczne dotyczące redukcji śladu węglowego. Jako że komunikacja zbiorowa generuje mniejszy ślad węglowy na osobę w porównaniu do transportu indywidualnego (\citealt{Szymalski_Bukowicka_2022}, tab. 4 s. 80), to ułatwienie planowania podróży zbiorowym transportem regionalnym powinno przełożyć się na większe zainteresowanie tą formą transportu, a w konsekwencji na ograniczenie szkodliwych emisji. Z racji tego, rządy wielu krajów, a w szczególności krajów Unii Europejskiej będą przychylnie patrzeć na wszelkie inicjatywy promujące transport zbiorowy. Z dużą pewnością sprawia, że otoczenie prawne tego przedsięwzięcia powinno być przyjazne i stabilne, co dobrze rokuje dla przedsięwzięcia.
\begin{quote}\textit{„Wraz z rosnącym znaczeniem ochrony środowiska, oszczędzenia energi i redukcji śladu węglowego, zapewnienie zrównoważonego transportu stało się kluczowym wyzwaniem.”}
    %"With the increasing prominence of environmental protection, energy conservation, and carbon reduction issues [[ 1 ]], ensuring sustainable transportation [[ 3 ]] has become a critical and challenging goal."
\citep{Chung2023-ih} \end{quote}

Kolejną zaletą tej działalności jest to, że transport zbiorowy jest jest bardziej atrakcyjny dla ludzi o niższych dochodach, ponieważ pasażerowie go wybierający często szukają oszczędności w porównaniu do transportu indywidualnego. \citep{Nimorakiotaki2020-bl} Z tego względu usprawnienie transportu zbiorowego na terenach mało zurbanizowanych, a więc właśnie terenach o statystycznie niższych dochodach na osboę może się okazać owocne biznesowo, ponieważ mogło by zagospodarować niezaspokojone potrzeby mieszkańców tych obszarów.