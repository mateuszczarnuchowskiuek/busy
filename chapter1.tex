\chapter{Wprowadzenie}
\label{chap:pierwszy}

%(zobacz rys. \ref{fig:logo})
%\begin{figure}[ht]
%	\centering
%	\includegraphics[width=50mm]{images/pipp.png}
%	\caption{Prototyp loga platformy PIPP}
%	\caption*{Źródło: opracowanie własne.}
%    \label{fig:logo}
%\end{figure}

Poniższy tekst prezentuje projekt przedsięwzięcia e-biznesowego w postacu platformy integrującej prywatnych przewoźników.
Nie są oni bowiem obecnie zintegrowani w takim stopniu jak komunikacja miejska czy koleje, co sprawa, że planowanie podróży z ich udziałem jest znacząco mniej wygodne. Projekt przewiduje stworzenie platformy pozwalającej pasażerom na wyszukiwanie i planowanie tras z możliwością łatwego zakupu cyfrowych biletów współpracujących przewoźników wewnątrz platformy. Z drugiej strony dla przewoźników będzie ona źródłem wielu cennych danych analitycznych pozwalających im optymalizować trasy i kursy, a w konsekwencji zwiększać efektywność i zyski. Obecnie bowem ze względu na rozdrobnienie rynku przewozowego takie dane trudno pozyskać.
\begin{quote}
    \textit{„Wyraźnie brakuje wyczerpujących i bogatych w aktualne dane opracowań dotyczących szerzej pojętej tematyki autobusowych przewozów pozamiejskich. Jedną z głównych tego przyczyn jest postępujące rozdrobnienie rynku przewozowego, wywołujące trudności m.in. w prostej inwentaryzacji danych służących dalszym analizom.”} \citep{Wolański_Mrozowski_Pieróg_2016}
\end{quote}
Zdecydowanym atutem tego przedsięwzięcia jest także to, że wpisuje się ono w obecne światowe trendy polityczne dotyczące redukcji śladu węglowego. Jako że komunikacja zbiorowa generuje mniejszy ślad węglowy na osobę w porównaniu do transportu indywidualnego (\citealt{Szymalski_Bukowicka_2022}, tab. 4 s. 80), to ułatwienie planowania podróży transportem regionalnym powinno przełożyć się na większe zainteresowanie tą formą transportu, a w konsekwencji na ograniczenie szkodliwych emisji. Z racji tego, rządy wielu krajów, a w szczególności krajów Unii Europejskiej będą przychylnie patrzeć na wszelkie inicjatywy promujące transport zbiorowy. 
\begin{quote}\textit{„Wraz z rosnącym znaczeniem ochrony środowiska, oszczędzenia energi i redukcji śladu węglowego, zapewnienie zrównoważonego transportu stało się kluczowym wyzwaniem.”}
    %"With the increasing prominence of environmental protection, energy conservation, and carbon reduction issues [[ 1 ]], ensuring sustainable transportation [[ 3 ]] has become a critical and challenging goal."
(\citealt{Chung2023-ih}, s. 1) \end{quote}
Z dużą pewnością sprawia, że otoczenie prawne tego przedsięwzięcia powinno być przyjazne i stabilne, co dobrze rokuje dla przedsięwzięcia.

Kolejną zaletą tej działalności jest to, że transport publiczny jest biznesem odpornym na recesje, bo to właśnie podczas recesji zwiększa się udział ludzi korzystających z transportu publicznego, ponieważ pasażerowie wybierając transport zbiorowy szukają oszczędności w porównaniu do transportu indywidualnego.

Do tego dzięki zebranym danym platforma będzie w stanie zaoferować lokalnym przewoźnikom lepsze planowanie kursów i tras dzięki czemu będą oni w stanie zwiększyć swoją efektywność i zwiększyć zyski, co powinno się przełożyć na chęć ich udziału w tym przedsięwzięciu, a im więcej przewoźników nawiąże współpracę z platformą, tym system będzie miał więcej do zaoferowania dzięki efektowi sieci. Zbudujemy zatem silną pozycję na rynku co pozwoli nam na łatwą monetyzację.