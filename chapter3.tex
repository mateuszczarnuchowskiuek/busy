\chapter{Krytyczne czynniki sukcesu}
\label{chap:trzeci}

\section{Kluczowe atrybuty pełnionych usług}
\subsection{Aktualność i dokładność informacji}
Najważniejszym czynnikiem sukcesu jest dostęp aplikacji do aktualnych oraz dokładnych planów jazd pomniejszych przewoźników. Aplikacja musi przesyłać powiadomienia użytkownikom na temat zmian planu jazdy (opóźnień, przekierowań, etc.), zmian cen czy pojazdów. Dodatkowo, aplikacja musi zapewnić możliwość wystawiania opinii kierowcom oraz czarnej listy pasażerów ze strony przewoźników.

\subsection{Intuicyjność aplikacji}
By zagwarantować wymagania aktualności i dokładności informacji, aplikacja musi być prosta w obsłudze. Priorytetem jest zagwarantowanie najwyższej jakości UX zarówno dla użytkowników-pasażerów poszukujących połączeń między lokalizacjami, jak i użytkowników-przewoźników, wykonujących przejazdy oraz deklarujących plany podróży oraz ceny swoich usług.

\subsection{Dostępność aplikacji}
Aplikacja powinna być dostępna na wszystkich popularnych platformach - iOS, Android oraz w wersji webowej na komputerach personalnych. W przypadku urządzeń mobilnych, aplikacja musi być dostępna z poziomu App Store, Google Play oraz bezpośrednio do pobrania z oficjalnej strony, w formacie .apk dla urządzeń z systemem Android. 

\subsection{Szeroka dostępność usług}
Aplikacja musi być dostępna w możliwie największej ilości miejscowości. Priorytyzowane muszą być mniejsze miejscowości o wysokim ruchu drogowym - są to lokalizacje o największej potencjalnej bazie klientów względem populacji.

\subsection{Możliwość opłaty przejazdów}
Użytkownicy muszą mieć możliwość opłacania przejazdów z poziomu aplikacji - aplikacja powinna być zintegrowana z odpowiednią bramką płatniczą, by móc bezpiecznie procesować płatności. Po uiszczeniu opłaty, użytkownik-pasażer musi mieć możliwość wyświetlenia biletu, a użytkownik-przewoźnik zeskanowania go za pomocą aparatu w telefonie.

\subsection{Bezpieczeństwo danych}
Aplikacja musi spełniać wymagania RODO, szczególnie biorąc pod uwagę wrażliwą naturę przechowywanych danych, takich jak lokalizacja czy środki płatnicze.

\subsection{Personalizacja aplikacji}
Użytkownicy, zarówno pasażerowie, jak i użytkownicy, muszą mieć możliwość personalizacji aplikacji - zaczynając od dostosowywania interfejsu 