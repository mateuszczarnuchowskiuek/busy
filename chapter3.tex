\chapter{Krytyczne czynniki sukcesu}
\label{chap:trzeci}

\section{Kluczowe atrybuty pełnionych usług}
\subsection{Aktualność i dokładność informacji}
Najważniejszym czynnikiem sukcesu jest dostęp aplikacji do aktualnych oraz dokładnych planów jazd pomniejszych przewoźników. Aplikacja musi przesyłać powiadomienia użytkownikom na temat zmian planu jazdy (opóźnień, przekierowań, etc.), zmian cen czy pojazdów. Dodatkowo, aplikacja musi zapewnić możliwość wystawiania opinii kierowcom oraz czarnej listy pasażerów ze strony przewoźników.

\subsection{Intuicyjność aplikacji}
By zagwarantować wymagania aktualności i dokładności informacji, aplikacja musi być prosta w obsłudze. Priorytetem jest zagwarantowanie najwyższej jakości UX zarówno dla użytkowników-pasażerów poszukujących połączeń między lokalizacjami, jak i użytkowników-przewoźników, wykonujących przejazdy oraz deklarujących plany podróży oraz ceny swoich usług.

\subsection{Dostępność aplikacji}
Aplikacja powinna być dostępna na wszystkich popularnych platformach - iOS, Android oraz w wersji webowej na komputerach personalnych. W przypadku urządzeń mobilnych, aplikacja musi być dostępna z poziomu App Store, Google Play oraz bezpośrednio do pobrania z oficjalnej strony, w formacie .apk dla urządzeń z systemem Android. 

\subsection{Szeroka dostępność usług}
Aplikacja musi być dostępna w możliwie największej ilości miejscowości. Priorytyzowane muszą być mniejsze miejscowości o wysokim ruchu drogowym - są to lokalizacje o największej potencjalnej bazie klientów względem populacji.

\subsection{Możliwość opłaty przejazdów}
Użytkownicy muszą mieć możliwość opłacania przejazdów z poziomu aplikacji - aplikacja powinna być zintegrowana z odpowiednią bramką płatniczą, by móc bezpiecznie procesować płatności. Po uiszczeniu opłaty, użytkownik-pasażer musi mieć możliwość wyświetlenia biletu, a użytkownik-przewoźnik zeskanowania go za pomocą aparatu w telefonie.

\subsection{Bezpieczeństwo danych}
Aplikacja musi spełniać wymagania RODO, szczególnie biorąc pod uwagę wrażliwą naturę przechowywanych danych, takich jak lokalizacja czy środki płatnicze.

\subsection{Personalizacja aplikacji}
Użytkownicy, zarówno pasażerowie, jak i użytkownicy, muszą mieć możliwość personalizacji aplikacji - zaczynając od dostosowywania interfejsu, przez tworzenie własnego profilu, po oznaczanie ulubionych tras lub przewoźników jako ulubione.

\subsection{Wielojęzykowość}
W celu trafienia do większej ilości klientów, aplikacja musi mieć alternatywne wersje językowe. Języki europejskie oraz te należące do mniejszości etnicznych o znacznej populacji na terenie Polski stanowią priorytet.

\section{Marketing}
\subsection{Rozpoznawalność marki}
Aplikacja musi mieć prostą do zapamiętania oraz wypowiedzenia nazwę, zarówno dla osób polskojęzycznych jak i innego pochodzenia. Szata graficzna musi być przyjemna dla oka w celu rozpoznawalności oraz lepszego UX.

\subsection{Selekcja miejsc kampanii reklamowych}
Kampanie reklamowe muszą być przeprowadzane przede wszystkim na pomniejszych przystankach oraz w miejscowościach poza większymi miastami, aby trafiać zarówno do pasażerów, jak i przewoźników. Biorąc pod uwagę segment rynku, do którego skierowana jest aplikacja, opłacanie kampanii marketingowych na większych węzłach komunikacyjnych takich jak np. MDA Kraków może zostać pominięte.

\subsection{Marketing zagraniczny}
Marketing prowadzony musi być w wielu językach, szczególnie europejskich, oraz skupiać się na zachęcaniu klientów zagranicznych do zwiedzania pomniejszych miejscowości na terenie Polski.

\subsection{Promocje}
W miarę rozwoju przedsiębiorstwa, wprowadzenie kodów rabatowych finansowanych przez samo przedsiębiorstwo może być skutecznym sposobem na zachęcenie większej ilości użytkowników do korzystania z aplikacji.

\section{Wsparcie}
\subsection{Wsparcie techniczne}
Aplikacja musi być wyposażona w możliwość zgłaszania ewentualnych błędów i musi być często aktualizowana.

\subsection{Wsparcie klienta}
Linia telefoniczna z wsparciem klienta musi być dostępna przez całą dobę w razie nagłych przypadków czy potrzeby udzielenia informacji klientom, w języku Polskim oraz wyżej wspomnianych, prioretytowych językach zagranicznych.

\subsection{Wsparcie prawne}
Na wypadek nagłej potrzeby, przedsiębiorstwo musi mieć dostęp do wysokiej jakości wsparcia prawnego.